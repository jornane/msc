\chapter{Introduction}
\label{chp:introduction} 

The internet today is a very important communication network.
Nearly everybody is connected to it, many are dependent on it,
 and the possibilities are virtually endless.
The applications range from looking at pictures of cats, to high intensity stock trading and remote surgeries;
however, not all applications are well intended towards other internet users.
Criminals write malware that will send spam and spy on people's computers to look for credit card information.
Activists use botnets to bombard sites they don't like, in order to make them unavailable.

Internet providers want to block clients that participate in harmful activity,
 but this is not a simple task.
The amount of traffic that passes through a back bone router is huge,
 and special provisions need to be in place to log this.
Cisco is a large supplier of router hardware, and has introduced the NetFlow format~\cite{cisco0netflow}.
NetFlow data does need to be stored an processed, for which multiple solutions are available~\cite{zaharia2010spark,Morken352472}.

Current software for analysing NetFlow logs is based on counting flows.
A relatively high number of flows in a short period of time is reason for alarm,
 and a high number of flows for a single IP address may be a reason to investigate.
The challenge is that a high number of flows does not necessarily indicate a problem.
% which may lead to false positives, which may lead to the raising of the minimal amount of flows for alarm.


\section{Anomaly detection}
An anomaly is a non-conforming pattern~\cite{Chandola:2009:ADS:1541880.1541882}.
There are multiple methods to detect anomalous traffic.
%Different devices on a network generate traffic for many different reasons.
%Some traffic is benign, some traffic is malicious.
%Some examples of are: \emph{web browsing}, \emph{e-mail}, \emph{software updates}, \emph{torrents}, \emph{VPN tunnels}, \emph{SSH tunnels}, \emph{port scans}, \emph{spamming}, \emph{(D)DoS attacks} and \emph{infecting other hosts}.
%However, when observing network traffic, the only kinds of anomalies that can be observed from first glance are
It is possible to detect high or low resource consumption~\cite{lan2003effect}, standards-conforming or non-standards-conforming packets~\cite{john2008detection}, or asymmetric traffic which should be symmetric~\cite{kreibich2005using}.
However, these anomalies do not necessary indicate malicious activity.
Non-standard packets may be sent between hosts running legacy and/or proprietary software,
asymmetric traffic may be caused by network scanning tools, improperly configured firewalls or IP spoofing,
and high-rate traffic can just as well indicate a popular server, as an attack~\cite{gao2006differentiating}.

%This thesis introduces the SpreadRank algorithm,
% which will provide additional metrics to judge an IP address on.
%It will use NetFlow data to determine if traffic is repeated by other hosts,
% in other words, when a host receives a connection, it will see if this host starts initiating the same kind of connection to other hosts.
This thesis will focus on spreading of traffic as an anomaly, as malicious traffic is intended to spread as far as possible,
 while bona fide traffic will often remain between two hosts.
To illustrate: a criminal who created malware that steals credit cards wants it installed on as many computers as possible.
The owner of a botnet wants as many computers as possible to join it.
An end user sending an e-mail or doing online banking, however, will want this information to be only available to the intended recipients.


\section{Graph processing}
Graph processing is often associated with data mining, due to Facebook actively working on the technology to mine social data~\cite{ching2013scaling},
 though it has a lot of different uses.
Some examples of graph processing in real-world applications are navigation (roads as edges), financial (currency flow), internet search (Google pagerank~\cite{page1999pagerank}) and rumour spreading~\cite{nekovee2007theory}.
This thesis will use graphs to process network data from NetFlow, which is a graph that shows information flow.

The NetFlow data will be provided by UNINETT, which administers the core routers in the Norwegian academic network.
Backbone NetFlow traffic is typically hundreds of megabytes to gigabytes of flow data per day.
By observing such a large network, it is possible to look for patterns.
Observing individual network nodes may give some indication of the type of traffic being sent,
 though observing the interaction between many different hosts can show information that is not visible by only observing a few hosts.


\section{Privacy}
This thesis is based on sampled NetFlow data from December 2013, made available by UNINETT.
NetFlow data contains, among others, source IP, destination IP, protocol, and time, but no payload information.
Sampled NetFlow is a NetFlow log where not everything is logged, UNINETT logs about~1~\% of flows, this is done for performance and storage reasons.
Even though no payload information was made available for this study, payload information can be helpful when identifying traffic.
There are known techniques for using payload information without infringing on privacy~\cite{parekh2006privacy}.


%\section{Previous work}
%Earlier research has focussed on the MapReduce model for processing graph data~\cite{Morken352472}.
%
%
%A similar study compares computer viruses to biological viruses~\cite{pastor2001epidemic},
% it focuses on the spread/recovery rates of computer viruses.
%The study does not focus on graph based systems
%This study will not focus on virus infections, but it will look at the spreading of traffic on the internet.
%

% "Hadoop~\cite{Borthakur:2011:AHG:1989323.1989438}"


