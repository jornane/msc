\chapter{DOSRank}
\label{chp:dosrank}

To test Giraph, two simple algorithms, DOSRank and ReverseDOSRank are created, which count the amount of respectively incoming and outgoing edges from or to a vertex.
This proof of concept will count the amount of flows per \gls{service}.
It can be used to detect if a machine is executing or suffering from a \gls{DoS} attack,
 as these are typically identified by a large number of flows.
These two algorithms will count respectively the amount of outgoing and the amount of incoming edges.
DOSRank~(\ref{eq:dosrank}), which will calculate outgoing edges (incoming connections) is trivial; it will set its value during the first superstep and then vote to halt.
ReverseDOSRank~(\ref{eq:reversedosrank}) is a bit more complex;
 since incoming edges are not visible in Giraph, the first superstep is used to send a message over every edge.
During the second superstep, every vertex will receive an amount of messages that is equal to its amount of incoming edges.


\begin{equation}
	\label{eq:dosrank}
	f(x) = N^{-}(x).
\end{equation}
\begin{equation}
	\label{eq:reversedosrank}
	f(y) = N^{+}(y).
\end{equation}

\begin{algorithm}[h]
	\caption{DOSRank}
	\label{alg:dosrank}
	\begin{verbatim}
	result = vertex.getNumEdges();
	vertex.voteToHalt();
	\end{verbatim}
\end{algorithm}
\begin{algorithm}[h]
	\caption{ReverseDOSRank}
	\label{alg:reversedosrank}
	\begin{verbatim}
	if superstep = 0 then
	begin
	    result = 0;
	    sendMessageToAllEdges();
	end
	else
	begin
	    while(messages.hasNext)
	    do
	        message.next();
	        result++;
	    end;
	    vertex.voteToHalt();
	end.
	\end{verbatim}
\end{algorithm}

\section{Expected results}
The algorithm will find IP addresses that are associated with large amounts of flows.
Opening a large amount of flows is different from sending large amounts of data,
 as sending a large amount of data is often a completely legitimate thing to do:
For example, sharing files via peer-to-peer protocols or serving large files over HTTP.
A large amount of flows; however, may be an indication that something is amiss.
It can indicate port scanning, or even targeted attacks like SYN flooding;
however, a large amount of flows can also simply indicate a very active or popular host.

\section{Purpose}
These algorithms are just a proof-of-concept to show that graph systems can be used for traffic analysis.
The algorithm itself is most likely easier to implement using other systems.
A simple MapReduce algorithm could probably yield the same results a smaller amount of time~\cite{Morken352472};
however, this algorithm does show that Giraph can be used for traffic analysis,
 and it opens the door for more complex algorithms, like SpreadRank.


